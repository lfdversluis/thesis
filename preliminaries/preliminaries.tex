\chapter{Preliminaries}
\label{chp:preliminaries}

This chapter introduces protocols, general concepts and terminology that is used throughout this thesis.

\section{BitTorrent protocol}
BitTorrent is one of the most popular protocols for peer-to-peer file sharing, generating between 43\% and 70\% of internet traffic per country \todo{cite to support this statement}.
Downloading files is done in a decentralized way, that means no central entity such as a server is needed.
By obtaining a torrent file, users can download files associated with this torrent using a program that is compatible with the BitTorrent protocol.\\

A torrent is a file that contains information about the files and \todo{add how a torrent file works, what it contains and how you are downloading e.g. connecting to peers.}\\
 
When sharing files by using the BitTorrent protocol, a user (or peer) that uploads i.e. provides a file to the BitTorrent network, is called a seeder.
A peer that is downloading a file of a seeder is called a leecher.
Any peer can be both a seeder and leecher at the same time, and join the network at any given time.\\

The ratio between the total data downloaded and uploaded is called the seeding ratio \cite{Cohen-bittorrent}.
The seeding ratio can be seen as an indication of the level of collaboration i.e. giving back resources to the network.\\

%Seeding can be seen as an interaction between peers, where the seeder aids the leeching peer.
%By utilizing the seeders upload bandwidth, the leeching peer can use his download bandwidth to download a file.
%While there is a clear incentive for the leecher by downloading the desired file, there is none for the seeder.
%Especially since the leecher has a little chance of becoming also be a seeder for the original seeder \cite{Lai-Incentives}.\\

%Having peers actively and persistently contribute to the network will increase the network's health which in turn provides several benefits for all peers.
%A more healthy network results in a higher availability of seeders and results in high download speeds.
%It has been shown that private communities i.e. "darknets" where the seeding ratio is high, provides better download conditions \cite{meulpolder-privatecommunities}.
%In these private communities, trackers i.e. central components introduce peers to each other using the Tit-for-tat approach \cite{cohen-titfortat}.
%The Tit-for-Tat approach is aiding peers who have aided you in the past,
%The absence of trackers, which is often the case in public networks, results in free-riding \cite{Adar-Freeriding}.
%A free-riding peer does not or gives little back to the network while receiving all benefits i.e. download without any restriction.
%\todo{Explain the optimistic chocking approach.}
%BitTorrent applies a variation of the Tit-for-tat strategy, optimistic chocking, to combat this problem.
%The Tit-for-Tat strategy is to only provide help to peers that return this help.
%However, it has been shown that this approach is not effective in battling abuse \cite{Pouwelse-tribler}.

\section{Asynchronous programming}
\label{sec:async-programming}

When programming asynchronously the function that is being called (callee) by a caller returns a so-called deferred.
A deferred is a place holder for the actual value that this function will return once the callee has computed it.
In normal (synchronous) programming, when calling a function the caller is waiting for the callee to provide an answer. 
Making a synchronous call in a single or multi-threaded environment may not be efficient.
If the callee takes a long time to compute and return the answer, the caller thread will be idle for that duration.
When using deferreds, the callee will simply return a Deferred and starts the computation.
The caller will resume its other operations and when the callee is done, it will fire a callback event, notifying that the deferred has either resolved into an answer or an error.
By attaching a callback and an errback, the caller can handle the case of a success and failure respectively.\\

One of the dangers of asynchronous programming is that during the callee's computation, the caller will also continue.
This may result in the caller changing values, the callee is dependent on.

\todo{more stuff here}

\section{Dispersy}
Dispersy is a middleware system for data dissemination in  network.
In Dispersy one can define communities which can send messages to each other.
Dispersy does not rely on a central component with the exception of bootstrap servers.
Dispersy is used heavily within Tribler and is maintained by the Tribler organisation.
The main use of Dispersy is the possibility of sending and receiving data to and from peers in the network \cite{zeilemaker-dispersy}. \todo{more explanation needed.}