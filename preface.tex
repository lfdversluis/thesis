\chapter*{Preface}
\addcontentsline{toc}{chapter}{Preface}
The increasing surveillance by governments who seek to get a hold on Internet is showing its mark more.
In countries such as China and X, where censorship is common practice shows that there is an (increasing) need for anonymity online.

The huge increase in Bitcoin adoption, 
due to the increase in uncertainty in the security of traditional currency 
after the financial crisis in 2008, 
is only rivaled by the speculation of the enormous possibilities 
of the underlying technology, the block chain.
The block chain is the first technology, seen with real world adoption,
that allows to register transactions without a trusted third party.
The block chain has several limitations in scalabillity
that will limit Bitcoin in fully replacing traditional currency.
It also limits block chain as a scalable base for a large scale reputation system,
similar to a digital currency, with vast amounts of transactions.
These developments provide motivation to research the properties and possibilities of a new distributed data structure:
MultiChain.

\vspace{1\baselineskip}

\noindent
Acknowledges go here.

\vspace{1\baselineskip}

\noindent
Laurens Freydis Dene Versluis

\vspace{1\baselineskip}

\noindent
Delft, The Netherlands

\noindent
\today
