% This document provides the style to be used for a MSc Thesis at the
% Parallel and Distributed Systems group
\documentclass[11pt,twoside,a4paper,openright]{report}

% use babel for proper hyphenation
\usepackage[british]{babel}
% Graphics: like the DUT logo on the front cover
\usepackage[dvips]{graphicx}
%Enables [H] for figures.
\usepackage{float}
% FONT: times
\usepackage{times}
% for url's use "\url{http://www.google.com/}"
\usepackage{url}
% To allow margin adjustments
\usepackage{changepage}
% To allow listings.
\usepackage{listings}
% To inserts to do's
\usepackage{todonotes}
% Boxed verbatim for experiments
\usepackage{fancyvrb}
% Side by side packages
%\usepackage{subfigure}
% sub caption package
\usepackage{subcaption}
% bib in order of appearance
\bibliographystyle{ieeetr}

\newsavebox{\FVerbBox}
\newenvironment{FVerbatim}
 {\VerbatimEnvironment
  \begin{center}
  \begin{lrbox}{\FVerbBox}
  \begin{BVerbatim}}
 {\end{BVerbatim}
  \end{lrbox}
  \fbox{\usebox{\FVerbBox}}
  \end{center}}

\begin{document}

%%%%%%%%%%%%%%%%%%%%%%%%%%%%%%%%%%%%%%%%%%%%%%%%%%%%%%%%%%%%%%%%%%%%%%%%%%%%%%%
\hoffset=1.63cm
\oddsidemargin=0in
\evensidemargin=0in
\textwidth=5in

%%%%%%%%%%%%%%%%%%%%%%%%%%%%%%%%%%%%%%%%%%%%%%%%%%%%%%%%%%%%%%%%%%%%%%%%%%%%%%%
% No identing
\setlength\parindent{0pt}

\pagestyle{empty}

% FRONTCOVER
\begin{titlepage}

\null\vfill

\begin{center}
\LARGE{Refactoring complex open-source projects: Improve performance while reducing code complexity}
\end{center}

\vspace{1.5cm}

\begin{center}
Laurens F. D. Versluis\\
L.F.D.Versluis@student.tudelft.nl
\end{center}

\vfill

\centering
\begin{figure}[!b]
\captionsetup[subfigure]{labelformat=empty}
\begin{subfigure}{0.4\textwidth}
\centering
\includegraphics[height=3cm]{pics/triblerlogo}
\caption{}
\end{subfigure}%
\begin{subfigure}{0.6\textwidth}
\centering
\includegraphics[height=3cm]{pics/dslogo}
\caption{}
\end{subfigure}%
\end{figure}

\begin{figure}[!b]
\centering
\includegraphics[height=2cm]{pics/TUDLogo}
\end{figure}


\vspace{2.0cm}

\end{titlepage}


% EMPTY PAGE
% \cleardoublepage

\pagestyle{plain}

% TITLE PAGE: page i (hidden)
\begin{titlepage}

  \begin{center}
  \null\vfill
    \begin{center}
    \LARGE{Title again.}
    \end{center}

    \vspace{3cm}

    \begin{large}
    Master's Thesis in Computer Science
    \end{large}

    \vspace{1.5cm}

    \begin{normalsize}
	Distributed Systems group\\
    Faculty of Electrical Engineering, Mathematics, and Computer Science\\
    Delft University of Technology
    \end{normalsize}

    \vspace{2.0cm}

    \begin{normalsize}
    Laurens F. D. Versluis
    \end{normalsize}

    \vspace{1.0cm}

    % <MM> DD, YYYY
    \today

  \vfill
  \end{center}

\end{titlepage}



% GRADUATION DATA AND ABSTRACT: pages ii and iii (hidden)
%De aankondiging bevat de spreker, titel, plaats, datum en tijd, samenstelling van de afstudeercommissie en een korte samenvatting (maximaal 25 regels).
\thispagestyle{empty}

\noindent \textbf{Author}\\
\begin{tabular}{l}
Laurens F. D. Versluis\\
\\
\end{tabular}\\
\noindent \textbf{Title}\\
\begin{tabular}{l}
Resolving bottlenecks in highly complex distributed systems\\
\\
\end{tabular}\\
\noindent \textbf{MSc presentation}\\
\begin{tabular}{l}
% <MM> DD, YYYY (like \today)
August 29, 2016
\\
\end{tabular}

\vspace{1.1cm}

\noindent \textbf{Graduation Committee}\\
\begin{tabular}{ll}
% The order of listing the names: Graduation prof, supervisor(s), others ordered by title + alphabetical
%examples:
%prof. dr. ir. H. J. Sips (chair) & Delft University of Technology \\
%ir. dr. D. H. J. Epema           & Delft University of Technology \\
dr. ir. J. A. Pouwelse            & Delft University of Technology \\
dr. ir. M. Zuniga                 & Delft University of Technology \\
dr. ir. A. Bozzon                 & Delft University of Technology \\

\end{tabular}

\begin{abstract} %de abstract bevat alleen een korte samenvatting van de inhoud van het onderzoek
Performance is a make-or-break quality for software.
When making changes it is essential to ensure no \emph{performance regression} has occurred i.e. the program performs more slowly or consumes more resources than previous versions.
Tribler is the result of ten years scientific research in complex distributed systems.
Over the course of years Tribler's performance has fallen below acceptable user experience levels, mainly because there is a lack of \emph{software performance engineering}.

In this work, we lay the foundations for a regression testing systems that allows developers to continuously monitor the metrics that are having the most impact on the performance of Tribler. 
Utilization of this system gave us a deep insight in the greatest bottleneck of the performance of Tribler: synchronous disk operations. Resolution of this bottleneck includes a major refactoring effort of the message synchronization system Dispersy. 
Implementation of a novel, non-blocking disk operation framework allowed us to increase the responsiveness of Tribler with 200\%.
\end{abstract}


\end{abstract}

% \clearpage



\pagenumbering{roman}
\setcounter{page}{4}

% EMPTY PAGE: page iv
% \cleardoublepage

% OPTIONAL QUOTATION: page v
%\include{quotation}
% EMPTY PAGE: page vi
%\cleardoublepage

% PREFACE: page v
\chapter*{Preface}
\addcontentsline{toc}{chapter}{Preface}
This thesis presents the work I have conducted in the last ten months.
During this period, I have had the pleasure of meeting new people who made my time much more enjoyable.
I would like to thank some people who contributed to this thesis.
Firstly, I would like to thank Johan Pouwelse for offering me this thesis and providing feedback on my work.
Secondly, I would like to thank Elric Milon who shared his knowledge of Tribler, Python, Git and Emacs with me as well as maintaining and updating our infrastructure from which everyone profited.
Thirdly, I am grateful to Pim Otte and Selina Versluis for providing feedback on my work.
I would also like to thank both Martijn and Hans, who I have spent the most time with in the MSc lab and for occasionally helping me when struggling with Linux.
Next, I would like to thank Ernst, Niels, Paul and everyone else of the Tribler team for providing feedback and suggestions.
Last but not least, I would like to thank my friends and family for their support, motivation and encouragement throughout my thesis.

\vspace{1\baselineskip}

\noindent
Laurens Freydis Dene Versluis

\vspace{1\baselineskip}

\noindent
Delft, The Netherlands

\noindent
\today
% EMPTY PAGE: page vi
% \clearpage

% TABLE OF CONTENTS: starting at page vii
\setcounter{tocdepth}{2}
\tableofcontents

% \cleardoublepage

\pagenumbering{arabic}
\setcounter{page}{1}

% CHAPTERS ... For instance: History/Prior Work, Design/Implementation, Experiments
% INTRODUCTION
\chapter{Introduction}
\label{chp:introduction}

% Performance evaluation of software architectures:  http://dl.acm.org/citation.cfm?id=287353

% Experience with Performance Testing of Software Systems: Issues, an Approach, and Case Study: http://search.proquest.com/docview/195577700?pq-origsite=gscholar

% The Future of Software Performance Engineering: http://ieeexplore.ieee.org/xpls/abs_all.jsp?arnumber=4221619&tag=1

% Performance Regression Testing Target Prioritization via Performance Risk Analysis: http://opera.ucsd.edu/paper/icse14-perfscope.pdf

% 

Performance is a make-or-break quality for software.
In today's society we expect manufacturers to progress in their expertise and their products to advance.
We take for granted that new cars have a larger radius, are more powerful, less pollute and more safe with each iteration of design.
In contrast, software does not has this image; software updates breaking functionalities, adding severe performance issues or increasing the complexity of the user interface are common.
The Facebook mobile Android application is a prime example of this, having more than one billion installs and being in development for years by a complete development department still receives updates which causes performance issues for users, see Figure~\ref{fig:facebook_bad_reviews}.
Just as with their cars users expect software performance to advance or at least not become \emph{worse} in terms of performance. 

\begin{figure}[h]
	\centering
	\includegraphics[width=\linewidth]{introduction/images/bad_reviews.png}
	\caption{Bad reviews showing the Facebook Android application introducing issues with updates.}
	\label{fig:facebook_bad_reviews}
\end{figure}

Performance can be divided in two dimensions: \emph{responsiveness} and \emph{scalability}.
Responsiveness is the ability of a system to meet the requirements for response time or throughput.
The response time can be measured by how fast a system can respond to an event, where throughput can be measured by how many events can be processed in a set amount of time.
Scalability is the ability of a system to meet the required response time or throughput when faced with a growing demand of its software functions.

Performance plays an important role in both industrial and academic areas.
Smith et al. have shown that the cost of a software product is determined more by how well it achieves its objectives for quality attributes such as performance than by its functionality \cite{smith2003software}.
For instance, an increase of 500 milliseconds latency in Google's search results could cause 20\% traffic loss \cite{mayer2009search}.  
The deterioration of performance introduced by changes is often referred to as \emph{performance regression}.
Performance regression can lead to several undesired consequences such as damaged customer relationships as the software does not meet its required performance.
Huang et al. provide the example of an e-commerce website that saw an increase of 2000\% in their page loading times because of an update to the underlying database engine \cite{huang2014performance}.
These situations can lead to lost revenue and possibly missed market windows.
Other consequences of performance failures may express themselves in lost productivity for users, increased costs, failures on deployment or even abandonment of projects \cite{woodside2007future, williams1998performance}.

To inspect whether the performance of a system has decreased, performance regression testing can be applied.
Using this method, the system is tested for performance regression under various loads \cite{woodside2007future}.
Traditional software development focusses on correctness, causing regression testing to be deferred to a late stage in the development cycle, if applied at all.
%This approach is often called the \enquote{fix-it-later} approach.
To illustrate the varying amounts of regression testing appliances, Huang et al. mention the performing testing interval of MySQL, Linux and Chrome.
These projects apply performance tests every release, every week and every four revisions, respectively.
Once performance regression is detected, developers have to spend extra efforts determining what causes said regression, especially when a lot of changes have been applied to the code base since the last measurement \cite{huang2014performance}.
Performance tests should be executed as frequently as possible, ideally per change made by developers.
However, as some tests take hours or even weeks, this approach is not always feasible.

\emph{Software performance engineering} (SPE) is the discipline concerned with constructing software systems that meet performance objectives \cite{smith2003best}.
It prescribes principles for creating responsive software, methods to obtain performance specifications and offers guidelines for the types of evaluations to be conducted at each development stage.
SPE features two general approaches \cite{woodside2007future} where the first approach is purely measurement based.
This characterizes itself by performing actions late in the development cycle such as  applying regression tests, diagnosis and tuning, when the system can be run and measured in real-time.
The second approach features a model-based style.
Using this approach, performance models are created at the early stages of development, influencing the architecture and design of the system to meet performance requirements.

\todo{Uitleggen dat performance in centralized architecturen makkelijker is te tunen dan een decentraal systeem + plaatje met verschil}.

\begin{figure}[!h]
	\centering
	\includegraphics[width=\linewidth]{introduction/images/tribler_screenshot.png}
	\caption{Screenshot of Tribler v6.5.2.}
	\label{fig:tribler_screenshot}
\end{figure}

Tribler is the result of ten years of scientific research in the field of decentralized systems and online cooperation.
Over 100 scientific publications have Tribler in their foundation.
Tribler is completely open-source and can be downloaded from the Tribler website\footnote{\url{https://www.tribler.org}}.
Tribler's interface is visible in Figure~\ref{fig:tribler_screenshot}.\\
Over the course of more than a decade Tribler has gained a tremendous amount of attention in both media and academia.
It has been downloaded approximately 1.8 million times \cite{github2016releases} and has more than two thousand monthly active users.
This makes Tribler one of the research projects that allow researchers to run experimental code \enquote{in the wild} on a large scale.

One of Tribler's unique features is that it allows users to discover and exchange data in a complete decentralized way.
This introduces additional challenges compared to centralized solutions.

Using a centralized structure, heavy computations can be offloaded to servers which are often more powerful and more responsive than a consumer grade computer.
Once the result has been computed, it can be communicated back to the user at only the cost of the communication. 
It has been demonstrated that for devices with a finite power supply such as smartphones, this technique can be applied to save energy or to boost performance \cite{kumar2010cloud, kemp2010cuckoo}.

Decentralized systems such as Tribler do not have such servers present that can be offloaded to, restricting the area of computation to the device itself.
It is thus essential that the software performs well on heterogeneous devices, possibly facing challenging network conditions.

To enhance privacy and anonymity, support for anonymous downloads was introduced in 2014 by R. Plak \cite{plak2014anonymous} and R. Tanaskoski \cite{tanaskoski2014anonymous}.
In 2015, the support for anonymous seeding of torrents using Tor-like hidden services was added by R. Ruigrok \cite{ruigrok2015bittorrent}.
A trade-off has to be made by Tribler between the desired performance and the level of anonymity provided, as any additional layer of privacy comes with an increased number of cryptographic operations.

Not only anonymity impacts the overall performance of Tribler: architectural flaws introduced in the past have led to a decrease in performance today.
This manifests itself in users frequently reporting high system load, a non-responsive Graphical User Interface and low download speed.
Furthermore, Tribler is plagued with a high number of disk operations which has been known since 2013 \cite{pouwelse2014reduce}. 

The focus of this thesis is to improve Tribler's performance by making use of software performance engineering techniques in the late stages of the development cycle with a particular focus on the performance of disk operations and software regression testing.

The rest of this thesis is structured as follows.
Chapter~\ref{chp:problem-description} provides the problem description and the research questions this thesis attempts to answer.
Chapter~\ref{cpt:pythons_thread_model} explains the python threading model and why we observe performance issues with Tribler.
Chapter~\ref{cpt:experiments} presents experimental results of this work.
Finally, Chapter~\ref{cpt:conclusion_and_future_work} concludes this thesis and provides future work.

%Problem description
\chapter{Preliminaries}
\label{chp:preliminaries}

This chapter introduces the BitTorrent protocol and general concepts and terminology that is used throughout this thesis.

\section{BitTorrent protocol}
BitTorrent is one of the most popular protocols for peer-to-peer file sharing, generating between 43\% and 70\% of internet traffic per country \todo{cite to support this statement}.
Downloading files is done in a decentralized way, that means no central component i.e. server is needed.
By obtaining a torrent file, users can download files associated with this torrent using a program that is compatible with the BitTorrent protocol.\\

A torrent is a file that contains information about the files and \todo{add how a torrent file works, what it contains and how you are downloading e.g. connecting to peers.}\\
 
When sharing files by using the BitTorrent protocol, a user (or peer) that uploads i.e. provides a file to the BitTorrent network, is called a seeder.
A peer that is downloading a file of a seeder is called a leecher.
Any peer can be both a seeder and leecher at the same time, and join the network at any given time.\\

The ratio between the total data downloaded and uploaded is called the seeding ratio \cite{Cohen-bittorrent}.
The seeding ratio can be seen as an indication of the level of collaboration i.e. giving back resources to the network.\\

%Seeding can be seen as an interaction between peers, where the seeder aids the leeching peer.
%By utilizing the seeders upload bandwidth, the leeching peer can use his download bandwidth to download a file.
%While there is a clear incentive for the leecher by downloading the desired file, there is none for the seeder.
%Especially since the leecher has a little chance of becoming also be a seeder for the original seeder \cite{Lai-Incentives}.\\

%Having peers actively and persistently contribute to the network will increase the network's health which in turn provides several benefits for all peers.
%A more healthy network results in a higher availability of seeders and results in high download speeds.
%It has been shown that private communities i.e. "darknets" where the seeding ratio is high, provides better download conditions \cite{meulpolder-privatecommunities}.
%In these private communities, trackers i.e. central components introduce peers to each other using the Tit-for-tat approach \cite{cohen-titfortat}.
%The Tit-for-Tat approach is aiding peers who have aided you in the past,
%The absence of trackers, which is often the case in public networks, results in free-riding \cite{Adar-Freeriding}.
%A free-riding peer does not or gives little back to the network while receiving all benefits i.e. download without any restriction.
%\todo{Explain the optimistic chocking approach.}
%BitTorrent applies a variation of the Tit-for-tat strategy, optimistic chocking, to combat this problem.
%The Tit-for-Tat strategy is to only provide help to peers that return this help.
%However, it has been shown that this approach is not effective in battling abuse \cite{Pouwelse-tribler}.

\section{Asynchronous programming}
\label{sec:async-programming}

When programming asynchronously the function that is being called (callee) by a caller returns a so-called Deferred in the Python programming language.
The Deferred is a place holder for the actual value that this function will return once the callee has computed it.
In normal (synchronous) programming, when calling a function the caller is waiting for the callee to provide an answer. 
Making a synchronous call in a multi-threaded environment where one thread calls the function of another may not be efficient.
If the callee takes a long time to compute the return value, the caller thread will be idle for that duration.
When using Deferreds, the callee will simply return a Deferred and starts the computation.
The caller will resume its other operations and when the callee is done, it will fire a callback event, notifying that the Deferred has either resolved into an answer or an error.
By attaching a callback and an errback, the caller can handle the case of a success and failure respectively.\\

One of the dangers of asynchronous programming is that during the callee's computation, the caller will also continue.
This may result in the caller changing values, the callee is dependant on.
When programming asynchronously, 

% \section{Dispersy}
% Dispersy is a middleware system for data dissemination in a network.
% In Dispersy one can define communities which can send messages to each other.
% Dispersy does not rely on a central component with the exception of bootstrap servers.
% Dispersy is used heavily within Tribler and is maintained by the Tribler organisation.
% The main use of Dispersy is the possibility of sending and receiving data to and from peers in the network \cite{zeilemaker-dispersy}.\section{BitTorrent protocol}
When sharing files by using the BitTorrent protocol, a peer that uploads parts of a file to another peer, is called a seeder.
A peer that is downloading a file of a seeder is called a leecher.
Any peer can be both a seeder and leecher at the same time, and join the network at any given time.

The ratio between the total data downloaded and uploaded is called the seeding ratio \cite{Cohen-bittorrent}.
The seeding ratio can be seen as an indication of the level of collaboration i.e. giving back resources to the network.

Seeding can be seen as an interaction between peers, where the seeder aids the leeching peer.
By utilizing the seeders upload bandwidth, the leeching peer can use his download bandwidth to download a file.
While there is a clear incentive for the leecher by downloading the desired file, there is none for the seeder.
Especially since the leecher has a little chance of becoming also be a seeder for the original seeder \cite{Lai-Incentives}.

Having peers actively and persistently contribute to the network will increase the network's health which in turn provides several benefits for all peers.
A more healthy network results in a higher availability of seeders and results in high download speeds.
It has been shown that private communities where the seeding ratio is high, provides better download conditions \cite{meulpolder-privatecommunities}.
In these private communities, trackers i.e. central components introduce peers to each other using the Tit-for-tat approach \cite{cohen-titfortat}.
The Tit-for-Tat approach is aiding peers who have aided you in the past,
The absence of trackers, which is often the case in public networks, results in free-riding \cite{Adar-Freeriding}.
A free-riding peer does not or gives little back to the network while receiving all benefits i.e. download without any restriction.
\todo{Explain the optimistic chocking approach.}BitTorrent applies a variation of the Tit-for-tat strategy, optimistic chocking, to combat this problem.
The Tit-for-Tat strategy is to only provide help to peers that return this help.
However, it has been shown that this approach is not effective in battling abuse \cite{Pouwelse-tribler}.
\chapter{Problem description}
\label{chp:problem-description}

Tribler's goal is to offer a YouTube-like experience with similar performance and ease of use.
All Tribler's features are implemented in a completely decentralized manner, not relying on any centralized component.

Numerous initiatives exist around these goals of re-decentralisation and performance.
However, none of them gathered any significant usage compared to the social media usage levels. 
For instance, YouTube features one billion unique monthly users \cite{mainka2014government} and there are 1.8 billion monthly active Facebook users \cite{sharma2016strategies}.

The problem is that the performance, usability, and features offered by decentralised alternatives are inferior when compared to the experience offered by central solutions.
Creating academically pure self-organising systems such as Tribler has proven to be notoriously difficult.
For example, the extensive list of 194 projects which all aim to create an alternative Internet experience using decentralisations shows the amount of years spent and lines of code produced \cite{redecentralize2016alternative}.
Most of these projects are abandoned and few of them have actual real-world usage.

\begin{figure}[!h]
	\makebox[\textwidth][c]{\includegraphics[width=\linewidth]{problemDescription/images/roadmap}}
	\caption{Three of the six uncompleted Tribler roadmap items.}
	\label{fig:tribler_roadmap}
\end{figure}

The Tribler project created a roadmap -- available on its GitHub repository -- to offer the same service, features, user experience, and performance as the YouTube video-on-demand service.
However, especially the poor performance of Tribler is hampering wide-spread adoption and usage. Figure~\ref{fig:tribler_roadmap} shows three of the six main uncompleted roadmap items.

\section{Key Performance Optimizations}

Tribler can be seen as a large and complex distributed system.
Metrics from OpenHUB show that Tribler, along with its components, features more than 169 thousand lines of code, received contributions from 111 unique contributors and took approximately 44 years of effort \cite{openhub2016tribler}.

In large and complex systems there are likely to be many performance issue present, often referred to as \emph{bottlenecks}.
J. M. Juran's Pareto principle admonishes that one should "Concentrate on the vital few, not the trivial many" \cite{ammons2004finding}. This principle is also known as the 80/20 rule.
Concretely, this means that resolving the vital bottlenecks yields the best diminishing returns, even for large systems such as Tribler.
After careful scrutiny it was decided that the most vital bottleneck to address, within the context of a nine month thesis, is Tribler's database I/O.

\section{Addressing Blocking I/O}
\label{sct:triblers_database_dependency}

The problem we address within this thesis is the underlying reason for poor performance and unacceptable user experience. 
Measurements dating back from 2013 indicate that Tribler's performance is I/O-bound \cite{pouwelse2014reduce}.
Especially with slow hard disks, but also with fast SSD storage the main performance bottleneck seems to be around database access.
With our focus on the fundamental issue we believe we can make a significant step forward in making decentralized technology able to compete with centralized solutions on large-scale usage.

\begin{figure}[!h]
	\makebox[\textwidth][c]{\includegraphics[width=\linewidth]{problemDescription/images/dispersy_database_schema}}
	\caption{The database schema of Dispersy, (source: Johan Pouwelse, 2013) \cite{pouwelse2013documentation}.}
	\label{fig:dispersy_database_schema}
\end{figure}

All information within Tribler is stored in a database for persistence and ease of use.
Information about the network i.e. peers, messages and authentication is stored in a separate database managed by the Distributed Permission System (Dispersy).
Dispersy is an elastic database system written in the Python programming language and uses SQLite as its underlying database engine.
It lies at the heart of Tribler, providing the means to discover peers and content in a decentralized way while offering security and anonymity.

Dispersy is fully decentralized with the exception of bootstrap servers.
It can run on systems with a large number of nodes, without any sever architecture needed \cite{dispersy2016dispersy, zeilemaker2013dispersy}.
All nodes perform the same algorithmic procedures and tasks and do not differentiate between any node i.e. all nodes are equal.

Furthermore, Dispersy provides one-to-one and one-to-many data dissemination mechanisms to forward data to nodes.
Eventually, all data will reach all nodes in the network, overcoming challenging network conditions.
The current overview of the database structure is presented in Figure~\ref{fig:dispersy_database_schema} (Johan Pouwelse, 2013).

These databases are becoming a key performance bottleneck \cite{pouwelse2014reduce}.
Back in May 2013 measurements indicated that Tribler read and wrote 660 Megabytes per hour to and from disk.
The next measurement in April 2014 showed this number was somewhat reduced to 623.
In May 2014 efforts were made to reduce this enormous amount of I/O; by batching database statements the number dropped to 538 megabytes per hour.

So far this metric has only been measured sporadically by hand, running Tribler for an arbitrarily amount of time and check the amount of I/O by using htop\footnote{\url{http://hisham.hm/htop/}}.
htop produces an overview similar to Figure \ref{fig:iotop_tribler_april_2014} (Johan Pouwelse, 2014).
Measurements to observe to which extent Dispersy is responsible for these numbers were never conducted, however it is strongly suspected by the Tribler developers that Dispersy is responsible for most of it.
Since 2014 no work or measurements have been done related to this issue.

\begin{figure}[!h]
	\makebox[\textwidth][c]{\includegraphics[width=0.9\paperwidth]{iointribler/images/iotop}}
	\caption{A screenshot of htop showing Tribler's I/O, (source: Johan Pouwelse, 2014 \cite{pouwelse2014reduce}).}
	\label{fig:iotop_tribler_april_2014}
\end{figure}

\subsection{Blocking I/O}
One of the main causes of Tribler's performance issues can be explained by the blocking behaviour of I/O.
Currently, Dispersy is deeply embedded into Tribler, running on the same (main) thread Tribler is running on.
Tribler, just like Dispersy, is written in the Python programming language.
In Python, a thread performing an I/O operation will block, causing all operations on the thread to suspend.
This means whenever Tribler or Dispersy performs I/O all functions of Tribler and Dispersy halt.
With the enormous amount of I/O Tribler is performing, this forms a huge limiting factor on the responsiveness and therefore performance of Tribler.

When a thread suspends, other threads can take over and perform operations, yet besides the main thread there are only two additional threads in Tribler: the Graphical User Interface (GUI) thread and the Dispersy endpoint thread.
As the name implies, the GUI is running on the GUI thread as the framework Tribler currently uses requires this.
Ironically, the Dispersy endpoint thread was introduced because of the blocking I/O behaviour.
Under heavy load, Dispersy drops incoming packets because it cannot keep up.
Processing packets is done on the main thread and as this thread frequently blocks, the buffers overflow causing packet loss.
These two threads do not saturate the available processing time offered by the main thread blocking, leading to wasting valuable CPU cycles.

Furthermore, Tribler has seen several changes to its code base including the addition of the MultiChain: Tribler's own Blockchain-like structure \cite{norberhuis2015multichain}.
This feature heavily relies on its database to store blocks and other information about the user and other peers.
Moreover, the MultiChain makes use of its own database rather than Dispersy's.
Norberhuis points out: \enquote{The information is stored in two places within Tribler and this could be eliminated. It would reduce the disk footprint and the amount of read/write transactions as only one database would have to be maintained. The I/O ineractions[sic] are a problem according to Tribler maintainers.} \cite{norberhuis2015multichain}, yet numbers on how much I/O the MultiChain generates are not presented.
This makes it hard to estimate Tribler's current I/O rates.

What's more, a feature called \enquote{credit mining} is currently in development that will also interact with the database of Tribler.
There are no metrics on the current situation of Tribler and it's hard if not impossible to estimate the impact of any addition to come.
Currently, there is a lack of insight in these metrics, or a lack of \emph{software performance engineering}, causing the exact extent of the problem to be unknown.

%\section{Asynchronous I/O}
%\label{sec:async-programming}

\section{A Lack of Software Performance Engineering}

\begin{figure}[!h]
	\makebox[\textwidth][c]{\includegraphics[width=\linewidth]{problemDescription/images/commits_openhub.png}}
	\caption{The distribution of commits on Tribler (source: OpenHUB, 2016 \cite{openhub2016tribler}).}
	\label{fig:commits_openhub}
\end{figure}

Nowadays, programs are evolving at a rapid speed and Tribler is no exception. 
Since 2005 Tribler is under continuous development, over seventeen thousands commits have been made, spanning more than a decade \cite{openhub2016tribler}.
The distribution of these commits can be seen in Figure~\ref{fig:commits_openhub}.
Code commits to fix bugs, refactor code, enhance or add functionality are pushed to the code repository of Tribler at a frequent rate.
It is important that software performance engineering is a part of this process: performance should not get compromised by changes, if any it should improve.
Naturally trade-offs can be made performance wise, but it should be done with a clear understanding of the consequences.

For the past three years, attempts have been made to monitor the performance, yet with little success.
The first attempt was to create probes using Systemtap\footnote{\url{https://sourceware.org/systemtap/}}.
System tap is a tool for \enquote{instrumenting the Linux kernel for analyzing performance and functional problems}, (Jacob et al. 2008) \cite{jacob2008systemtap}.
While some success was reported, this system is no longer being maintained nor functional.
After this, code was added to the Dispersy code base to track and log if a function was running longer than a fixed amount of time.
While this implementation does provide some insight, its workings are crude and only covers some of the functions present.
For instance, it cannot handle asynchronous constructions.
Tribler did not have any observation system integrated, leaving the development team in the dark regarding its performance.

It is apparent that there is a lack of software performance engineering in the development cycle of Tribler.
Performance has never been one of the priorities in Tribler's lifetime: only 6\% of all tickets on GitHub are (indirectly) related to performance based on their content.
To ensure performance will no longer degrade, realistic benchmarks need to be developed which Tribler can be tested with.
Changes can then be compared against the current code base, tracking important performance statistics such as the amount of I/O, run time of functions, throughput and responsiveness.
These benchmarks can then be integrated into a regression testing system which can be integrated in our Jenkins continuous integration system.
Using Jenkins, performance regression tests can run on every proposed change and at predetermined moments, allowing for a continuous updated overview of Tribler's performance metrics.

% https://github.com/Tribler/tribler/issues/15
% https://github.com/Tribler/tribler/issues/77
% experiment framework: https://github.com/Tribler/tribler/issues/114
% latency grpahs? https://github.com/Tribler/tribler/issues/119
% speed of streaming: https://github.com/Tribler/tribler/issues/134
% of course: https://github.com/Tribler/tribler/issues/3
% nightly runs: https://github.com/Tribler/tribler/issues/184
% extreme cpu ticket: https://github.com/Tribler/tribler/issues/197
% slow startup time ticket: https://github.com/Tribler/tribler/issues/255

\section{Realistic Benchmarks}
Directly related to the performance problem is the benchmark problem.
In order to improve user performance we require making assumptions about realistic use cases.
Each user has different usage patterns, hardware and network conditions, all affecting performance.
Creating several benchmarks testing realistic scenario's is required to accurately tune the system for real world usage.
At the same time, a benchmark cannot consume too much time.
Benchmarking is by nature time consuming \cite{huang2014performance}, however running long regression tests per commit will severely strain the development speed.

Therefore, it is important to create a reference benchmark which has a close resemblance to real world usage without consuming too much time.

\section{Objective and Research Questions}
\label{chp2:sct:objectives-research-questions}
The objective of this thesis is to improve the performance and responsiveness of Tribler and to introduce a regression testing system.
The verification of the performance regression testing system is done by focussing on removing Tribler's biggest bottleneck present: blocking database I/O.
By resolving this bottleneck, important metrics tracked by this regression testing system should show positive changes, indicating improvement.

The research presented in this thesis was carried out in cooperation with the Tribler team. 
The Tribler team consists of both staff members of the Technical University of Delft as well as Bachelor and Master students.
Based on the objectives of Tribler, this thesis aims to answer the main research question formulated below.\\

\textbf{Main Research Question:} How can we improve Tribler's performance, responsiveness and throughput?\\

To answer this main research question, we have defined three research questions below. Each of these research questions will be justified as why they contribute to the main research question.\\

\textbf{Research Question 1:} Can a system such as Tribler benefit from asynchrony?\\

To improve performance and responsiveness, parts of Tribler can be rewritten to become asynchronous.
By performing tasks asynchronously the performance and responsiveness of a program can improve.
However, an asynchronous approach can have its drawbacks. 
One of these drawbacks is that it requires a different mindset for the programmers as the whole call chain and structure of a program becomes different.
Identifying these drawbacks and deciding if the benefits outweigh the costs is necessary to prevent the current state from worsening. \\

\noindent
\textbf{Research Question 2:} How do we resolve Tribler's blocking database I/O problem?\\

As database I/O is currently the main bottleneck, we need to resolve it.
Tribler already has a framework integrated that is especially designed to handle I/O in a non-blocking way.
We require an approach that prevents us from reinventing the wheel while still solve the bottleneck at hand in an adequate manner.
Additionally, we need to make sure the order of database operations does not change as this may lead to inconsistencies.

Careful scrutiny is required to look at available solutions and their pros and cons.
A decision can then be made on the best course of action. \\

\noindent
\textbf{Research Question 3:} How do we incorporate software performance engineering into Tribler's development process to gain insight into performance statistics?\\

Currently, not a single developer has insight into how well Tribler performs and what impact changes have on Tribler in its current state.
To be able to conclude performance changes do not negatively impact the performance of Tribler, we can apply software performance engineering.
Software performance engineering focuses on introducing performance regression tests and benchmarks into the development cycle.
By making use of these regression tests, Tribler developers finally get insight into vital metrics which is desperately needed.\\


\section{Main Contributions}
The main contributions of this thesis are as follows.
First, we elaborate on the subject of multitasking and parallelization in the context of the Python programming language and provide arguments where asynchronous programming is preferred over synchronous programming, using Tribler as a case study.
We then resolve the vital I/O bottleneck currently present in Tribler using asynchrony and a multi-threaded approach.
This is done by introducing the Storm database framework into Tribler and creating a database manager with an asynchronous, non-blocking yet serialized interface.
Next, we introduce software performance engineering into Tribler's development process by adding a regression testing system that benchmarks different versions of the same code base to gain insight into changes in performance metrics such as disk I/O.
Finally, experimental results and measurements will be provided to confirm the main goal of this thesis i.e. improving responsiveness, performance and throughput.

%Related work
\chapter{Related work}

Many attempts have been made to remove the GIL from Python to fully benefit from multiple CPU's, yet no one has ever succeeded in removing the GIL and meet the (hard) requirements for replacement \cite{python2015global}.

On of the most well-known alternative implementations of Python is Pypy.
It makes use of a Just-in-Time compiler to offer increased speed, reduced memory usage and support for stackless mode while providing a highly compatibility with existing python code.
Pypy's geometric average is 7.6 times faster than CPython (normal Python) \cite{pypy2016speed}.
While it has many popular libraries ported to be used with Pypy, it still lacks some common used packages.
Moreover, most of these libraries are not available on the official packaging repositories of Ubuntu and/or Debian.

Two other popular implementations are JPython and IronPython.
Both of these projects have removed the GIL and can fully exploit multiprocessor systems \cite{python2015global}.

JPython is a Python interpreter implemented in Java. It can be integrated in Java applications and allows python applications to be compiled into Java classes.
Using JPython, Python after compiled to java bytecode will run in the Jython virtual machine, giving full access to all Java APIs and classes \cite{jython2016why}.

IronPython does basically the same as JPython, compiling the source to in-memory bytecode and runs it on the Dynamic Language Runtime \cite{ironpython2014}.
It allows developers to run Python using the .NET framework.

To illustrate the attempts to remove the GIL are still on going, Larry Hastings presented ``The Gilectomy'' at PyCon 2016. He showed that removing the GIL is fairly easy, but has a huge negative impact on CPython's performance.
Additionally, he explained the reason why this performance loss was observed and names some methods that may make ``The Gilectomy'' a viable alternative to CPython.\\

Libraries and frameworks that introduce asynchrony are also available in large numbers.
Decorated Concurrency (DECO) uses the multiprocessing package of Python to parallelize functions using a ``concurrent'' decorator \cite{sherman2016deco}.
Similarly, the ``synchronized''  inserts synchronization events to automatically refactors assignments of the results of ``concurrent'' function calls to happen during synchronization events.

Twisted is an event-driven networking engine written in Python.
It allows for event-driven and asynchronous programming using deferreds.
Twisted has a custom event-loop called the reactor, on which tasks can be scheduled.
This reactor handles callbacks/errbacks fired by deferreds and contains many utility functions and classes to perform asynchronous and non-blocking calls.
Unlike many libraries and frameworks available, Twisted is available on the official repositories of Ubuntu and Debian and is a well tested, mature framework.

\todo{Describe asynchronous related work}
\todo{Describe Bottle neck search/find related work}
\todo{Describe connectable related work as in connectability in challenging peer to peer networks}
\todo{Describe refactoring processes and how to properly write code and test?}

%Experimentation
\chapter{Implementation and experiments}
TBD.
%Conclusion
\chapter{Conclusion and future work}
\label{cpt:conclusion_and_future_work}

Future work

\begin{itemize}
	\item Move storage to a per-community level. This will mean less flushing of data when sending items. In Tribler one only sends messages in the allchannel and to the channelcommunity (if you own a channel).
	\item A threadpool with sqlite3 multithreaded enabled? 
	\item Move database stuff into classes and use Storm's ORM. That auto-caches items, doesn't commit immediately.
	\item Extend the regression test system with more metrics? Memory, network etc. ?
	\item Scan for commits that look scary \cite{huang2014performance} and tests those. Will reduce amount of regression testing needed. Or use Jenkins pipeline to only run experiment when a value is set to true or tests pass?
\end{itemize}


% BIBLIOGRAPHY
%\bibliographystyle{bib/latex8}
\bibliography{bib/bibliography}

%\appendix

%\include{appendix_a}

\end{document}

