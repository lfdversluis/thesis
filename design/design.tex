\chapter{Design \& Implementation}
\label{cpt:design_implementation}

After the initial results and discussion of asynchrony in python, it is evident that asynchronous I/O is beneficial to Tribler's performance.
Since almost all of Tribler's I/O operations are database operations, it is key point to focus on.
This translates itself into making Dispersy's database I/O asynchronous and non-blocking.
Tribler has Twisted integrated for years, yet Dispersy has not seen any integration despite the decision to do so in 2014 \cite{pouwelse2013consider}.
To ensure a good foundation to build upon without reinventing the wheel, it is key to search for a framework that supports both SQLite (Dispersy's current database system) and asynchronous I/O using Twisted.
This framework will then become the basis of a new database manager.

\section{A new database framework}

\begin{table}[]
	\centering
	\caption{An overview which features each of the four frameworks support.}
	\label{table:database_frameworks_comparison}
	\begin{tabular}{l|c|c|c|c|}
		\cline{2-5}
		& \textbf{Twistar} & \textbf{Storm} & \textbf{Axiom} & \textbf{Alchimia} \\ \hline
	\multicolumn{1}{|p{4cm}|}{Available in the Debian \& Ubuntu repositories} 	& \xmark & \cmark & \cmark & \xmark \\ \hline
	\multicolumn{1}{|l|}{Allows \enquote{raw} queries} 							& \cmark & \cmark & \cmark & \cmark \\ \hline
	\multicolumn{1}{|l|}{Allows an ORM approach} 								& \cmark & \cmark & \cmark & \xmark \\ \hline
	\multicolumn{1}{|l|}{Framework is mature} 									& \cmark & \cmark & \cmark & \xmark \\ \hline
	\end{tabular}
\end{table}

With the recent addition of the MultiChain there are three distinct database files with three distinct database managers in the Tribler code base.
None these database managers are fully documented or tested.
A proper solution is to replace these three database managers with the new asynchronous one.
This will result in less code to maintain, all logic in one place and easier to cover with proper unit tests and documentation, yielding increased stability and speed, improved maintainability and enhances the productivity of developers.

After careful scrutiny, four database frameworks that offer integration with Twisted and SQLite were selected: Axiom, Storm, Alchemia and Twistar.
Next, they were compared on the possibility to use it as an object-relational mapper (ORM), the possibility to query the database using \enquote{raw} queries, its maturity and the availability in the official repositories of Ubuntu and Debian which is a must as Tribler is published on the official repositories as well.
The results of this comparison can be found in Table~\ref{table:database_frameworks_comparison}.

From this table it is clear that Twister and Alchimia are not good fits; neither of them are available in the official repositories of Ubuntu and Debian.
After comparing Axiom and Storm in better detail the final decision led us to choose Storm.
The Storm database framework which is developed by Canonical and is featured in several other products such as Launchpad \cite{canonical2011storm}, showing its real world use and maturity.
The Storm website features a rich tutorial and documentation section, superior to that of Axiom, where new developers joining Tribler will benefit from.
Additionally, all table creation and updates must explicitly be handled by the developer which is Tribler's and Dispersy's current approach.
As we favour this enforcement over automatically generated tables, Storm was chosen as the foundation of the new database manager: \enquote{StormDBManager}.

\section{Designing StormDBManager}

\begin{figure}[h]
	\makebox[\textwidth][c]{\includegraphics[width=\linewidth]{experimentation/diagrams/storm_db_worker.png}}
	\caption{An overview of the queueing mechanism of StormDBManager.}
	\label{fig:storm_db_worker}
\end{figure}

StormDBManager features a complete asynchronous, non-blocking yet serialized interface to handle database access.
Because Storm also features ORM support, this database manager can be the foundation for an ORM based approach in the future.

Since multi-threaded support is severely limited using SQLite, we decided to leverage the Twisted thread-pool to allocate a thread for a longer period of time to run a worker on.
This worker will be owned by the StormDBManager.
Using this approach, all database operations happen on the same thread but outside the Twisted main thread, guaranteeing I/O does not block it.
Furthermore, the StormDBManager guarantees that queries will be executed in the same order as they are scheduled, guaranteeing no conflicts in database fields can occur because of race conditions.
The system works as follows, visualized in Figure~\ref{fig:storm_db_worker}.
Fist, a Dispersy function calls the StormDBManager to run a query (1).
The StormDBManager generates a deferred and returns this to the caller (2).
Next, the StormDBManager queues a tuple of four elements (3):

\begin{enumerate}
	\item The function to be called, e.g. execute or fetchone.
	\item The arguments to be passed to the function i.e. the query.
	\item The keyword arguments to be passed to the function.
	\item A deferred to handle the response in an asynchronous way.
\end{enumerate}

Note that by using a thread-safe queue, all calls are scheduled in the same order as required, ensuring serialized behaviour.
The worker running on the thread waits blocking for new items to come, preventing the thread from dying.
Once it a tuple is available it fetches it (4).
It then executes the function (5) and calls the deferred's callback with the result (6).
After that, the worker proceeds to wait blocking for a new item, or executes the next tuple if present.
To make sure the worker can still commit or release the thread, two predetermined values can be queued upon which the worker will commit or shut down, respectively.

\section{Implementing StormDBManager}

As the new StormDBManager will start retuning deferreds, functions of Dispersy need to be able to coop with this new paradigm.
Every caller of this function will need to be transitively updated as well to handle the deferreds being returned.

\begin{figure}[h]
	\begin{subfigure}[b]{.5\linewidth}
		\lstinputlisting[caption={Foo synchronous},label={lst:foo_sync},language=Python]{design/listings/foo_sync.py}
	\end{subfigure}
	\begin{subfigure}[b]{.5\linewidth}
		\lstinputlisting[caption={Foo asynchronous},label={lst:foo_async},language=Python]{design/listings/foo_async.py}
	\end{subfigure}
	\caption*{Example of the same function synchronous and asynchronous.}
\end{figure}

To keep the amount of changes to a minimum we have made extensive use of the \enquote{inlineCallbacks}\footnote{\url{http://twistedmatrix.com/documents/current/api/twisted.internet.defer.inlineCallbacks.html}} decorator.
The inlineCallbacks decorator allows programmers to write asynchronous code in a synchronized manner.
To illustrate this in an example, consider the two code samples of the same function called \enquote{foo} in Listings \ref{lst:foo_sync} and \ref{lst:foo_async}.
The left listing shows the synchronous version of foo calling a function \enquote{bar} which for example performs a database query.
After refactoring bar to make use of the \enquote{StormDBManager} it will become asynchronous, returning a deferred.
To handle this, we can need to update foo to cope with this.
The right listing shows the refactored version of foo; it is decorated with the inlineCallbacks decorator and has now a \enquote{yield} statement in front of the bar function call.
Twisted automatically waits for bar's deferred to fire and then continues with the execution.

Consequentially, because of the inlineCallbacks decorator, foo is now an asynchronous function as well, returning a deferred whenever called.
As a result all functions that call foo needs to be updated transitively as well.

In total there are 129 function calls to Dispersy's database (excluding tests): 26 fetchall, 29 execute, 53 fetchone, 2 insert, 9 executescript, 10 executemany.
There functions were spread across X methods which all needed to be refactored transitively as with the example provided previously.
After Dispersy was fully refactored, 47 files were modified with 4605 lines of additions and 2003 of deletion.
To express the extent of this refactoring, approximately 90\% of all Dispersy functions were updated to handle the asynchrony introduced.
Naturally Tribler also required modifications; in total 106 files required modifications spanning 3572 additions and 1242 deletions to the code base.
Finally, an experiment framework called Gumby (see Section~\ref{sct:gumby_introduction}) required modifications as well, resulting more than eleven thousand modified lines of code spanning more than six months of work.



\todo{Replace 90\% with actual number.}