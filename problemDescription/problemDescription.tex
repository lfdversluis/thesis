\chapter{Problem description}
In many areas of the world, freedom of speech is not self-evident.
Those who wish to discuss and critique e.g. governments in these areas may turn to technologies such as Tor.
These technologies often employ strong encryption schemes and random routes to hide the user and ensure anonymity.
Most of these technologies use a centralized or semi-centralized structure.
Scalability and connectability are often problems that come with these technologies.

Tribler is a peer-to-peer BitTorrent client that attempts to fully decentralize downloading, uploading and streaming of content.
Being fully decentralized, Tribler scales well [CITE], however some problems reside.
In this chapter, several problems are described that Tribler is currently facing.


\section{Connectability}
One of the problems Tribler faces is the connectability of its users.
Users can be behind e.g. firewalls that prevent incoming or outgoing connections.
Currently, Tribler employs NAT and Firewall Traversal [CITE], yet the success of this method is limited.

\section{Anonymous download speed}
To allow for anonymous downloading, Tribler uses a Tor-like technique to create `anon tunnels'.
To increase download speed and prevent a bottleneck in the network hampering the overall download speed, several tunnels are built to download simultaneously from [CITE?].
Since this Tor-like technique includes cryptographic functions, having multiple tunnels running concurrently results in a lot of cryptographic calls being executed.
Stok et al. profiled Tribler using a gumby experiment. This experiment showed that a lot of processing time is spent on encrypting and decrypting packets [CITE].
On top of that, Tribler is written in the python language, which only utilizes one CPU core and features a global interpreter lock.
This interpreter lock prevents threads running separately to actually perform work concurrently.
As a result, the CPU core on which Tribler is running is usually at full capacity and thus being a bottleneck.
By offloading the encryption and decryption of packets to a separate core, the workload on the core on which Tribler is running reduces.
This should yield an increased download speed and better spread of workload overall.

\section{Package serialization}
One of the core components of Tribler is Dispersy. Dispersy is a fully decentralized elastic database capable of communicating packets and exchanging peers.
Both Dispersy and Tribler make use of python' built-in serialization function.
The profile ran by Stok et al. [CITE] indicated that a significant amount of CPU resources are spent on serializing data when sending or receiving UDP packets.
As previously explained, running multiple threads at once does not result in code being executed concurrently.
Cap'n Proto is a serialization framework that offloads the serialization code to C++, which bypasses the global interpreter lock.
This means that the other threads -- who are still affected by the global interpreter lock -- can immediately continue their work as the locks is released on offloading it to C++.

\section{Scalability}
TBD.