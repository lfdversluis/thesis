\chapter{Python's threading model}

Python threads are normal operating system (OS) threads, either POSIX or Windows threads.
They are fully managed by the operating system that hosts them.
In the Python programming language there is a Global Interpreter Lock (GIL).
This GIL ensures that only one thread can run in the python interpreter at once, i.e. a thread needs to hold the GIL in order to execute.
Once a thread is done executing or needs to wait it releases the GIL.
This gives way for cooperative multitasking as visualized in Figure X \todo{Make a picture like in understanding-the-python-gil}.
Other threads that are ready to execute can try to acquire the GIL, the thread that obtains the GIL is determined by the OS.

To make sure CPU heavy tasks do not hold the GIL indefinitely a simple check is built in that ``checks'' every thread once per 100 (default) ticks.
Ticks are loosely mapped to interpreter instructions and do not define a time unit.
\todo{Show code sample that takes 1 tick but takes long in time.}
When a check is run the following steps are executed:
\begin{itemize}
	\item The thread that holds the GIL resets its tick counter.
	\item If the current thread it the main thread, it runs the signal handlers.
	\item The thread releases the GIL.
	\item The thread tries to reacquire the GIL.
\end{itemize}



\section{Multi threaded programming performance}

\section{Asynchronous programming}
