\chapter{Conclusion and Future Work}
\label{cpt:conclusion_and_future_work}

This thesis aims to contribute to the goal of re-decentralisation of systems such as Tribler, in particular with a focus on providing a YouTube-like performance.
We have addressed Tribler's blocking database I/O, its main performance bottleneck, by integrating the Storm database framework into a new database manager: \enquote{StormDBManager}.
StormDBManager features a complete asynchronous, non-blocking interface for database access while still maintaining a serialized query execution strategy.
Furthermore we have provided deep insight into Tribler's and Dispersy's database usage, pinpointing functions that could be reviewed for query optimization.
Additionally we have created a regression testing system and prepared it to be integrated in our Jenkins continuous integration system to adopt software performance engineering in the development cycle, further maturing the project.
We have verified both the regression testing system and the resolving of the bottlenecks by providing experimental results.
We believe that with this performance boost and software performance engineering focus, we have contributed to Tribler's further years of research and strengthened Tribler's chances on becoming a decentralized alternative for YouTube-like streaming.

While we believe we made a significant step forward in both performance and software performance engineering, there are items left for future work.

As different platforms and operating systems may influence the performance of a program, it is useful to run regression tests on all platforms Tribler supports.
Deploying the regression test system on all platforms will ensure no regression occurs on one of them.

To remove the \enquote{raw queries} from all code bases, an object-relational mapping approach can be applied.
This will reduce the complexity of the system as all data will be contained in objects which are generally easier to modify and read from.
This will require extensive refactoring of Tribler and Dispersy's code bases.

To improve performance further, moving Tribler from Python2 to Python3 is another possibility.
As we have seen, the Global Interpreter Lock in Python 3.2 and onwards has been updated to handle I/O-bound threads better.

To increase the speed of processing database queries, it can be investigated if we can use SQLite's multi-threaded support to process more queries.
While SQLite developers themselves admit SQLite has minimal multithreaded support, it can still be investigated to what extent we can leverage its support.

Improving the code quality of Dispersy is another item that was mentioned in this thesis.
It was found that passing all tests in Dispersy does not provide any guarantees beyond a basic level of correctness.
Improving and adding tests is required to increase the stability, maintainability and correctness of Dispersy.

A final item that we would like to highlight as future work is running benchmarks in a closed environment, disconnected from the Internet.
When running Tribler and Dispersy, it will connect to the Internet which may influence performance metrics such as I/O rates and amount of packets obtained.
Creating a closed environment with local peers will ensure that only those peers can communicate to one another.
These peers can then be instrumented to behave in a predetermined manner.
This will increase both the reliability and accuracy of the measurements made.
