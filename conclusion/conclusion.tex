\chapter{Conclusion and future work}
\label{cpt:conclusion_and_future_work}

In this thesis we have improved Tribler's performance significantly by making use of software performance engineering.
We have addressed Tribler's blocking database I/O, its main performance bottleneck, by integrating the Storm database framework into a new database manager: \enquote{StormDBManager}.
StormDBManager features a complete asynchronous, non-blocking interface for database access while still maintaining a serialized query execution strategy.
Additionally we have created a regression test system to ensure software performance engineering becomes part of the Tribler development cycle, further maturing the project.
We have verified both the regression test system and the resolving of the bottlenecks by providing experimental results.
We believe that with this performance boost and software engineering focus, we have readied Tribler for further years of research and strengthened Tribler's chances on becoming a decentralized alternative for YouTube-like streaming.


Future work

\begin{itemize}
	\item Move storage to a per-community level. This will mean less flushing of data when sending items. In Tribler one only sends messages in the allchannel and to the channelcommunity (if you own a channel).
	\item A threadpool with sqlite3 multithreaded enabled? 
	\item Move database stuff into classes and use Storm's ORM. That auto-caches items, doesn't commit immediately.
	\item Extend the regression test system with more metrics? Memory, network etc. ?
	\item Scan for commits that look scary \cite{huang2014performance} and tests those. Will reduce amount of regression testing needed. Or use Jenkins pipeline to only run experiment when a value is set to true or tests pass?
	\item Benchmarks not dependent on the internet, such as seeders for a torrent.
\end{itemize}